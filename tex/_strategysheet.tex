\chapter{Problem Solving Strategies}


\section{General Advice and Basic Patters}

\begin{itemize}
    \item If you need to Minimize a Maximum or Maximize a Minimum, use Binary Search.
    \begin{enumerate}
        \item https://www.codechef.com/KH19MOS/problems/ANAJOBS
    \end{enumerate}
    \item If you want to use DP / Expectation to find something optimal, use Exchange Argument.
    \begin{enumerate}
        \item https://www.codechef.com/GWR17ROL/problems/KALADIN
    \end{enumerate}
\end{itemize}

\section{List of Algorithms and Ideas we know}

\begin{itemize}
    \item 2-SAT
\end{itemize}

\section{Standard Problems for Practice}

\subsection{Math and Geometry}

\begin{itemize}
    \item Substitutions can help reduce the dimentionality of the problem (Kernel trick).
    \begin{enumerate}
        \item https://codeforces.com/problemset/problem/1142/C: Make the Substitution $z_{i} = y_{i}^{2} - x_{i}$. Now the problem is just computing the upper convex hull.
    \end{enumerate}
\end{itemize}

\section{Trees and Graphs}

\begin{itemize}
    \item Substitutions can help reduce the dimentionality of the problem (Kernel trick).
    \begin{enumerate}
        \item https://codeforces.com/problemset/problem/1142/C: Nice use of LCA for finding.
    \end{enumerate}
\end{itemize}

\begin{itemize}
    \item When solving a question involving strictly increasing number sequence, it's always a good idea to replace the $a_i$ with $a_i - i$.
    \begin{enumerate}
        \item It changes the constraint from strictly increasing to non-decreasing (allowing the equality constraint), which is generally very useful
        \item It puts contiguous groups of elements together (example, if $4, 5, 6, 7$ appear together in this order, they'll all be changed to the same number)
        \item Example problems where this technique was used: Edu 87 E, AGC 48 C.
    \end{enumerate}
\end{itemize}

\begin{itemize}
    \item If a question requires you to use two data structures in parallel, do this technique
    \item Run the first DS independently of the other and memoize its results.
    \item Then run the second DS and use memoized results from the first one.
    \begin{enumerate}
        \item This relies on the fact that this technically is still *online* processing, but the computations for the independent (first) DS are made independent of the second one, so they can be done more easily
        \item Example question: USACO 2019 USO Platinum Valleys 
    \end{enumerate}
\end{itemize}
